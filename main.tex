%%%% CS553 Cryptography Term Paper TEMPLATE %%%%

%%%% 1. DOCUMENTCLASS %%%%
\documentclass[preprint]{transcrypto}
%%%% NOTES:
% - Change "submission" to "final" for final version
% - Add "spthm" for LNCS-like theorems


%%%% 2. PACKAGES %%%%
\usepackage{lipsum} % Example package -- can be removed


%%%% 3. AUTHOR, INSTITUTE %%%%
\author{Gagan deep singh\inst{1,2} \and John Doe\inst{1}}
\institute{
 Indian Institute of Technology, Bhilai, India 
  \and
  
}
%%%% NOTES:
% - We need a city name for indexation purpose, even if it is redundant
%   (eg: University of Atlantis, Atlantis, Atlantis)
% - \inst{} can be omitted if there is a single institute,
%   or exactly one institute per author


%%%% 4. TITLE %%%%
\title{KLIEN CIPHER}
%%%% NOTES:
% - If the title is too long, or includes special macro, please
%   provide a "running title" as optional argument: \title[Short]{Long}
% - You can provide an optional subtitle with \subtitle.

\begin{document}

\maketitle


%%%% 5. KEYWORDS %%%%
\keywords{Something \and Something else}


%%%% 6. ABSTRACT %%%%
\begin{abstract}
  In this paper we prove that the One-Time-Pad has perfect security.

\end{abstract}


%%%% 7. PAPER CONTENT %%%%
\section{Introduction}


\section{Key Schedule}
We call the original key as master key and denote it as $mk$. It will be of 64/80/96 bit for  KLIEN 64/80/96. As a result of key scheduling algorthm 8/10/12  subkeys with length same as master key will be generated for KLIEN 64/80/96. Without LOG we will talk about key scheduling of KLEIN 64 \\

\begin{itemize}

\item The first subkey $sk_{0}$ is same as the master key \\
	$sk_{0}$=$mk$.Each of the subsequent $sk_{i+1}$ will be derived from $sk_{i}$ as follows - \\

\item Denote $sk_{i}$ as a tuple of bytes - (x0 x1 x2 x3 x4 x5 x6 x7) \\
Divide the tuple into two equal parts and call them a and b 

a - (x0 x1 x2 x3) \\
b - (x4 x5 x6 x7) \\

\item Now Perform one byte left circular shift to both a and b

a' = (x1 x2 x3 x0) \\
b' = (x7 x4 x5 x6) \\

\item Swap a' and b' i.e a'' = b' and b''=a' \\

\item Now let a'' = (y0 y1 y2 y3) \\
and b'' = (z0 z1 z2 z3) \\

We will Xor the round counter i with 3rd byte of a'' and pass 2nd and 3rd byte of b'' through the KLIEN S-BOX and then a''|b'' will become the next subkey \\
$sk_{i+1}$ = (y0 y1 $ y2 \oplus R_i $ y3 z0 Sbox(z1) Sbox(z2) z4)\\
\end{itemize}

\section{Main Result}
\label{sec:main}

%%%% 8. BILBIOGRAPHY %%%%
\bibliographystyle{alpha}
\bibliography{abbrev3,crypto,biblio}
%%%% NOTES
% - Download abbrev3.bib and crypto.bib from https://cryptobib.di.ens.fr/
% - Use bilbio.bib for additional references not in the cryptobib database.
%   If possible, take them from DBLP.

\end{document}
