\section{Observations}

\begin{frame}
The DDT(Differential Distribution Table) of SBOX is written below:\\ \\
\begin{center}
\scalebox{0.7}{ 
\begin{tabular}{|c|c|c|c|c|c|c|c|c|c|c|c|c|c|c|c|c|}
\hline
$\Delta_{In}|\Delta_{Out}$ &0&1&2&3&4&5&6&7&8&9&10&11&12&13&14&15 \\
\hline
0&16& 0& 0& 0& 0& 0& 0& 0& 0& 0& 0& 0& 0& 0& 0& 0 \\
\hline
1&0& 0& 0& 4& 2& 0& 2& 0& 2& 0& 0& 2& 0& 0& 2& 2 \\
\hline
2&0& 0& 0& 0& 0& 4& 0& 0& 0& 0& 2& 2& 0& 4& 2& 2 \\
\hline
3&0& 4& 0& 2& 2& 0& 0& 0& 0& 0& 2& 0& 0& 2& 4& 0 \\
\hline
4&0& 2& 0& 2& 2& 0& 2& 0& 0& 2& 0& 2& 0& 2& 0& 2 \\
\hline
5&0& 0& 4& 0& 0& 2& 0& 2& 2& 0& 2& 4& 0& 0& 0& 0 \\
\hline
6&0& 2& 0& 0& 2& 0& 4& 0& 2& 0& 2& 2& 2& 0& 0& 0 \\
\hline
7&0& 0& 0& 0& 0& 2& 0& 2& 2& 2& 0& 0& 2& 0& 4& 2 \\
\hline
8&0& 2& 0& 0& 0& 2& 2& 2& 2& 2& 0& 2& 0& 0& 0& 2 \\
\hline
9&0& 0& 0& 0& 2& 0& 0& 2& 2& 0& 0& 2& 2& 2& 2& 2 \\
\hline
10&0& 0& 2& 2& 0& 2& 2& 0& 0& 0& 4& 0& 2& 0& 2& 0 \\
\hline
11&0& 2& 2& 0& 2& 4& 2& 0& 2& 2& 0& 0& 0& 0& 0& 0 \\
\hline
12&0& 0& 0& 0& 0& 0& 2& 2& 0& 2& 2& 0& 4& 2& 0& 2 \\
\hline
13&0& 0& 4& 2& 2& 0& 0& 0& 0& 2& 0& 0& 2& 2& 0& 2 \\
\hline
14&0& 2& 2& 4& 0& 0& 0& 4& 0& 2& 2& 0& 0& 0& 0& 0 \\
\hline
15&0& 2& 2& 0& 2& 0& 0& 2& 2& 2& 0& 0& 2& 2& 0& 0 \\
\hline
\end{tabular}
}
 \end{center}
 The maximum differential probability is KLEIN's SBOX is 4/16 i.e 1/4 and some of the transitions leading to it are $(\Delta_{In},\Delta_{Out})$= (1,4),(3,1) etc.\\
\end{frame}

\begin{frame}{Attacks on KLEIN cipher}
\textbf{Round Reduced Attack}\\
The weakness present in the Rotate Nibbles and Mix columns step is exploited here in this attack.\\
Firstly, a 6 round truncated differential distinguisher with $2^{-29}$ is made.Using this as base, an 8 Round distinguisher is constructed.One of the assumption is they will be having access to round wise outputs.\\
Lets have look at few terminologies used in this attack:\\
\begin{enumerate}
    \item $X_{i}$ : The input of the i-th round.
    \item $\Delta X_{i}$ : The input difference of the i-th round.
    \item $Y_{i}$ : The input of SubNibbles in the i-th round .
    \item $\Delta Y_{i}$ : The input difference of SubNibbles in the i-th round .
    \item $X_{i,j}$ : The j-th nibble of the $X_{i}$ , where j = 0, 1, ...15.
    \item $sk_{i}$ : The subkey of the i-th round.
    \item $X || Y$ : The the concatenation of X and Y.
\end{enumerate}
\end{frame}

\begin{frame}
The state of encryption: \\
\begin{center}
\begin{tabular}{|c|c|c|c|}
\hline
0&4&8&12\\
\hline
1&5&9&13\\
\hline
2&6&10&14 \\
\hline
3&7&11&15\\  
\hline
\end{tabular}
\end{center}
The number in box represents the Nibble numbers of the 64 bit data block.
\textbf{Some properties and observations of KLEIN cipher}\\
\textbf{Lemma1.} If a byte is of the form $0z$, where $z$ is a 4-bit string
with MSB bit as 0, then $0z$ multiply by $x$ is equal to $0z^{'}$, where $z ^{'}$
is a 4-bit string.\\
The following oservations are derived based on above lemma.\\ \\

\end{frame}

\begin{frame}
    \textbf{Observation 1.} 
\begin{math}
\begin{bmatrix}
2&3&1&1\\
1&2&3&1\\
1&1&2&3\\
3&1&1&2
\end{bmatrix}
\text{x}
\begin{bmatrix}
0z \\
00 \\
00 \\
00 
\end{bmatrix} = 
\begin{bmatrix}
0z^{'}_{1} \\
0z^{'}_{2} \\
0z^{'}_{3} \\
0z^{'}_{4} 
\end{bmatrix}
\end{math} if only if MSB $z$ is 0. \\ \\
\textbf{Observation 2.} 
\begin{math}
\begin{bmatrix}
2&3&1&1\\
1&2&3&1\\
1&1&2&3\\
3&1&1&2
\end{bmatrix}
\text{x}
\begin{bmatrix}
00 \\
00 \\
0z_{1} \\
0z_{2} 
\end{bmatrix} = 
\begin{bmatrix}
0z^{'}_{1} \\
0z^{'}_{2} \\
0z^{'}_{3} \\
0z^{'}_{4} 
\end{bmatrix}
\end{math} if only if MSB $z_{1}$ and $z_{2}$ is 0. \\ \\
\textbf{Observation 3}
\begin{math}
\begin{bmatrix}
2&3&1&1\\
1&2&3&1\\
1&1&2&3\\
3&1&1&2
\end{bmatrix}
\text{x}
\begin{bmatrix}
0z_{1} \\
0z_{2} \\
0z_{3} \\
0z_{4} 
\end{bmatrix} = 
\begin{bmatrix}
0z^{'}_{1} \\
0z^{'}_{2} \\
0z^{'}_{3} \\
0z^{'}_{4} 
\end{bmatrix}
\end{math}
 if only if MSB's of $z_{1},
z_{2} ,
z_{3} \;and\;
z_{4} $ are 0.\\
\end{frame}
\begin{frame}{Truncated Six Round Differential Distinguisher}
Based on the above observations, we show that if the input difference of 6-round KLEIN are all zero except the 13-th nibble, after encryption, the first and the third column of state matrix will stay 0 with the probability of $2^{-29}$.\\
This happens because the difference in column will not transfer to other column due to the RotateBits algorithm and above observations. \\ This is the reason for high probability 6-round differential distinguisher.
The structure of 6-round distinguisher is clearly displayed in Figure \ref{fig:6distinguisher}.\\
\end{frame}
\begin{frame}
    
    \begin{figure}
    \centering
    \includegraphics[width= 8cm, height= 12cm, keepaspectratio]{./../images/6roundattack.png}
    \caption{6 Round Distinguisher \cite{reduced_round}}
    \label{fig:6distinguisher}
\end{figure}
\end{frame}

\begin{frame}{Truncated Differential Analysis of 8-Round KLEIN-64}
The 8 round distinguisher is constructed by adding an extra layer at the top and bottom of the 6 round distinguisher.\\
As this is CPA(chosen plain text) attack, we choose plain text pairs such a way that input difference to the second round is all zero except the 13-th nibble. We actually obtain the required input pairs of first round by reverse tracing the single nibble that should be active at the end of the round.This process is also diplayed in Figure \ref{fig:8distinguisher} \\
\begin{figure}
    \centering
    \includegraphics[width= \textwidth]{images/4to1_diff.jpg}
    \caption{First Round of the 8 Round Distinguisher}
    \label{fig:8distinguisher}
\end{figure}\\ \\
\end{frame}
\begin{frame}
\textbf{ The Steps and Analysis Procedure}\\ \\
\textbf{Step1.} We will choose the input plaintexts in such a way that, all nibbles have some fixed values except four nibbles $X_{1,1} , X_{1,3} , X_{1,13} , X_{1,15}$. If we fix the values and change the values only in these 4 nibbles, that is called one structure. There are $2^{16}$ possible plain texts in one structure.\\
We can form nC2 i.e $(2^{16} \text{x} (2^{16} - 1))/2 = 2^{31}$ plain text pairs from those $2^{16} $ plain texts.\\ If we took m structures then we will have $2^{16}m$ plain texts and $2^{31}m$ pairs.\\ \\ \\ \\
\textbf{Step 2.} By Guessing the values(trying all possible values)  of the subkey nibbles $sk_{1,1} , sk_{1,3} , sk_{1,13} , sk_{1,15}$ we should make sure that $\Delta SubNibbles(X_{1,1} \oplus sk_{1,1} ) = 0e, \Delta SubNibbles(X_{1,3} \oplus sk_{1,3} ) = 09, \Delta SubNibbles(X_{1,13} \oplus sk_{1,13} ) =
0d\; and \;\Delta S(X_{1,15} \oplus sk_{1,15} ) = 0b.$ The probability of this is $2^{-16}$ as we are fixing values of 4 particular nibbles, so the expected number of confirming pairs is $2^{31} \;\text{x}\; m\; \text{x} \;2^{-16} = 2^{15} m$. 

\end{frame}
\begin{frame}
\textbf{Step 3.} Now take those remaining $2^{15}m$ pairs and encrypt them upto 8 rounds. Then verify the third and first columns output difference of $MC^{-1}$  to zero. If not, discard the key guess. The probability of this event to happen is $(2 ^{-16})^{2}$ so the number of confirming pairs would be $2^{15} \;*\; m \;*\; (2 ^{-16})^{2} = 2^{-17}m$. Now we should use meet in the middle technique. \\ \\ \\
\textbf{Step 4}. For the obtained confirming pairs from previous step,  guess the value of the subkey $sk_{9,j}$, j = 0, 1, \ldots, 7 to inverse the SubNibbles step and find input difference of 8-th round, i.e $\Delta X_{8,j}$ , j = 0, 1, \ldots, 7. \\ \\
\end{frame}

\begin{frame}
    \textbf{Step 5.} Now, reverse the MixColumns step i.e $MC^{-1}$ on $\Delta X_{8}$ and verify whether the first column difference is zero and also the MSB of second column should be all 1 or all 0. If it failed to satisfy the above condition then discard the key guess.The expected confirming pairs is $2^{17}\;*\;m\;*\;2^{-7} = 2^{-24}m$ as the probability of the above event is $2^{-7}$.\\ \\ \\
\textbf{Step 6.} Guess the remaining 16 bits key in the similar way by verifying third and fourth columns.\\ \\ \\
\textbf{Time and Space Complexity} \\
Step 2 takes $2^{16}*2^{16}/8$ one round 64 bit encryptions. In Step 3 it takes $2^{15}*2^{16}*2^{16}*7/8$ encryptions whereas in step 4 it takes $2^{-17}*2^{16}*2^{16}*2^{32}/8$ encryptions are required and step 6 takes about $2^{16}$ encryptions.So overall time complexity $2^{46.8}$ one round 64 bit encryptions.The data complexity is $2^{32}$ plaintexts. Memory complexity is $2^{32}$ 64bit states. 
\end{frame}

\begin{frame}{Integral Analysis}
	\textbf{Observations}
	\begin{enumerate}
		
		\item If we give $2^{32}$ different input values to the Rotate Nibble then after the Rotate Nibble and sub nibble operation and 3 rounds of KLIEN all the output nibbles are balanced \\ \\
		
		\item If in the input state $i_{th}$ nibble is active where i=0,1,2,3,12,13,14,15 then after 1 round of KLIEN and one add key and one sub nibble all $j_{th}$ nibbles are active where j=8,9,10,11,12,13,14,15 \\
		\begin{tabular}{|c|c|c|c|c|c|c|c|}
			\hline
			A & A & C & A & C & C & C & C \\
			\hline
			C & C & C & C & A & A & A & A\\
			\hline
		\end{tabular}
		$\xrightarrow{1.5round}$
		\begin{tabular}{|c|c|c|c|c|c|c|c|}
			\hline
			X & X & X & X & X & X & X & X \\
			\hline
			A & A & A & A & A & A & A & A\\
			\hline
		\end{tabular}\\
	\end{enumerate}
\end{frame}
\begin{frame}
	\textbf{Combining Obs. to Get 5 round Distinguisher}\\
	\begin{tabular}{|c|c|c|c|c|c|c|c|}
		\hline
		A & A & C & A & C & C & C & C \\
		\hline
		C & C & C & C & A & A & A & A\\
		\hline
	\end{tabular}\\
	\begin{center}$\downarrow 5 round$\\ \end{center}
	\begin{tabular}{|c|c|c|c|c|c|c|c|}
		\hline
		B & B & B & B & B & B & B & B \\
		\hline
		B & B & B & B & B & B & B & B\\
		\hline
	\end{tabular}
\end{frame}

\begin{frame}{ 7 round integral attack}
	
	\begin{itemize}
	\item We can take our 5 round distinguisher one step further as Round key addition does not change the Balance property. Let $Y_{j}$ be the input to 6th round Sbox of the  plaintext where j denotes the jth nibble. Then the xor sum of $Y_{j}$ over all plaintext is 0 \\
	
	\\
	Let $y_{j}$ be the jth nibble obtained after Reversing the sub bytes.\\ \\ Clearly $y_{j}$=$S^{-1}$ ($X_{j}\oplus sk_{7,j}) $ 
	\\ \\
	\item After Mix columns of 6th round we get the following relation \\ \\
	$Y_{4}||Y_{5}$ = $S^{-1}(R^{-1}(e.(y_{1}||y_{2})\oplus b.(y_{2}||y_{3})\oplus d.(y_{4}||y_{5})\oplus9.(y_{6}|| y_{7}))
	\oplus sk_{6,4} || sk_{6,5})$
	\\ \\
	\end{itemize}
\end{frame}
\begin{frame}
	\textbf{Analysis Procedure}
	\begin{itemize}
		\item We first get \textbf{5} sets of $2^{32}$ plaintexts. For each of these sets we do the following-
		\item Guess $sk_{7,0}$,$sk_{7,1}$,$sk_{7,2}$,$sk_{7,3}$ for each plaintext and obtain $e.(y_{1}||y_{2})\oplus b.(y_{2}||y_{3})$\\
		Let $u1$=$e.(y_{1}||y_{2})\oplus b.(y_{2}||y_{3})$ \\.
		Now we get $2^{24}$ different values of $(u1\oplus d.(y_{4}||y_{5})\oplus9.(y_{6}|| y_{7}))$ as each of the term is 8 bytes.
		
		\item Guess $sk_{7,4}$,$sk_{7,5}$ and obtain 
		$d.(y_{4}||y_{5})$\\
		Let $u2$=$u1 \oplus (y_{4}||y_{5})$  \\  
		Now are left with $2^{16}$ values of $(u2\oplus9.(y_{6}|| y_{7}))$ \\
	\end{itemize}
\end{frame}
\begin{frame}
	\begin{itemize}
	
		\item Guess $sk_{7,6}$,$sk_{7,7}$ and obtain 
		$d.(y_{6}||y_{7})$\\
		Let $u3$=$u2 \oplus (y_{6}||y_{7})$  \\  
		Now are left with $2^8$ values of $(u3)$ \\
		\item Now our equation becomes \\
		$Y_{4}||Y_{5}$ = $S^{-1}(R^{-1}(u3)
		\oplus sk_{6,4} || sk_{6,5})$\\
		\item We now guess $sk_{6,4}$ and $sk_{6,5}$ and then obtain the Sum of $Y_{4}||Y_{5}$. If its not 0 we discard our guess. Wrong key can also give 0  with 1/128 probability and therefore we used 5 sets
	\end{itemize}
\end{frame}

\begin{frame}{Practical Attack on 8 Rounds of KLEIN}
	\textbf{Observations}
	\begin{itemize}
		\item \textbf{Observation 1.} If the difference entering MixColumn is of the
form 0000000X where X represents a non-zero difference in \{1, . . . , 7\} then the output difference is of the form 0Y0Y0Y0Y, where the wildcard Y represents a non-zero difference. That is, higher nibbles remain free of difference.\\
\item \textbf{Observation 2.} If the difference entering MixColumn is of the
form 0X0X0X0X where the wildcard X represents a difference in \{0, . . . , 7\}, then the output difference is of the form 0Y0Y0Y0Y,where Y represents a possibly null difference. Furthermore, the
average number of non-zero Y’s is 3.75, as one can experimentally verify. For example, the input difference 04020405 leads to the output difference 0f090100.\\
	\end{itemize}
\end{frame}

\begin{frame}{Observations Cont'd}
	\begin{itemize}
		\item \textbf{Observation 3.} If the difference entering MixColumn is of the form 0X0X0X0X where the wildcard X represents a difference in {8, . . . , f}, then the output difference is of the form 0Y0Y0Y0Y, where Y represents a (possibly zero) difference. Furthermore, the average number of non-zero Y’s is 3.75. Note that, unlike Observation 2, an X cannot be zero. For example, the input difference 0c0a080f leads to the output difference 010f0708.\\
\item \textbf{Observation 4.} Given a random difference, KLEIN’s Sbox
returns a difference in {1, . . . , 7} with probability 7/15 approximates to $2^{−1.1}$, for a random input. If the difference is b or e, the probability is 3/4 approximates to $2^{−0.42}$.\\
	\end{itemize}
\end{frame}

\begin{frame}{Finding More Right Pairs with Neutral Bits}
	\begin{itemize}
		\item A bit is said to be neutral with respect to a given differential
(characteristic) when flipping this bit in an input conforming to the differential (characteristic) leads to a new input also conforming to that differential. 
\item In KLEIN, one can observe that the first two and last two input bytes in a plaintext block are neutral with respect to the first two rounds’ collection of characteristics. 
\item Therefore, for example, after a $2^{28}$ effort to find a pair
satisfying the 6-round differential, one can derive $2^{32}$ pairs for which the full differential is followed with probability $2^{23.26}$.\\
	\end{itemize}
\end{frame}

\begin{frame}{Key Recovery Of 8 Rounds}
	\begin{itemize}
		\item The attack exploits the invertibility of the final MixNibbles and RotateNibbles to determine the output differences of each nibble after the last SubNibbles.
        \item  With approximately $2^{34}$ encryptions, one can identify a conforming pair with high probability.
        \item Using neutral bits, one expects to produce approximately 8 other conforming pairs after $2^32$ trials. This is more than enough to identify with certainty 32 bits of the last subkey.
        \item Overall, the 64 bits of the last subkey (and thus of the original key) can be found with complexity below $2^{35}$ encryptions.
	\end{itemize}
\end{frame}

\begin{frame}{Expanding to 7 and 8 rounds}
	\begin{itemize}
		\item We observe that for a pair conforming to the 6-round
differential, the SubNibbles of round 7 has all higher nibbles
inactive. Therefore a 7-round distinguisher can be built
with the same $2^{28}$ observations data complexity.
\item In the eight-round attack one first collects approximately $2^{33.90}$ pairs, and records the ones that conform to the output difference as per our collection of characteristics.
\item One expects to record approximately 4 pairs satisfying the
difference by chance, and one conforming to the collection
of characteristics. The conforming pair can be identified
using the neutral bits.
	\end{itemize}
\end{frame}

\begin{frame}{Expanding to 7 and 8 rounds}
	\begin{itemize}
		\item We observe that for a pair conforming to the 6-round
differential, the SubNibbles of round 7 has all higher nibbles
inactive. Therefore a 7-round distinguisher can be built
with the same $2^{28}$ observations data complexity.
\item In the eight-round attack one first collects approximately $2^{33.90}$ pairs, and records the ones that conform to the output difference as per our collection of characteristics.
\item One expects to record approximately 4 pairs satisfying the
difference by chance, and one conforming to the collection
of characteristics. The conforming pair can be identified
using the neutral bits.
	\end{itemize}
\end{frame}

\begin{frame}{Resistance against Side Channel Attacks}
	\begin{itemize}
		\item KLEIN posses a highly balanced key schedule wrt its opposition against Key-Related attacks and the dexterity of the keys, secret sharing method is used for resistance to side-channel attacks In which a CRT Algorithm is implemented.
				\item The masking based on secret sharing increases the hardware overhead,but still promising because it was theoretically proven to be secure against DPA attacks. Differential power analysis is a side-channel attack involving mathematically analyzing power consumption measurements from a crypto-system. It makes the use of varying power consumption of microprocessorsor other hardwares while performing operations using secret keys.
	\end{itemize}
\end{frame}



\begin{frame}{Performance Evaluation On AVR microcontroller}
	\begin{itemize}
		\item \textbf {With Respect To Energy Consumption}
		\item \textbf {With Respect To Memory Efficiency}
		\item \textbf {With Respect To Security}
	\end{itemize}
\end{frame}

\begin{frame}{With Respect To Energy Consumption}
	\begin{itemize}
		\item To measure energy consumption, it is assumed that the energy per CPU cycle is fixed.This energy consumption includes the key scheduling and encryption.KLEIN is the best algorithm in comparison to other ciphers in aspect of energy consumption.
		\item We learnt many important factors that effect the energy consumption such as the performance of code depends on several specifications of a cipher such as the Type of instructions, Mode of operation, Structure, Number of loops,Number of Rounds etc.
		\item Another important factor is which kind of instruction.it is important factor to use correct instruction to write a code.
	\end{itemize}
\end{frame}

\begin{frame}{With Respect To Energy Consumption Cont'd}
	\begin{itemize}
    \begin{figure}
    \centering
    \includegraphics[width= 75 mm]{Pics/EC.png} 
    \caption{Energy consumption comparison of focused ciphers}
    \label{fig:6distinguisher}
\end{figure} 
	\end{itemize}
\end{frame}

\begin{frame}{With Respect To Memory Efficiency}
	\begin{itemize}
		\item The memory usage of various lightweight algorithms is com-pared in Figure below which shows the percentage of memory used for each cipher. Analyzed results clearly state that KLEIN uses longer Flash memory space than the rest because the assembly code size of this algorithm is more than others Where as the percentage of SRAM usage for KLEIN cipher is not high as other ciphers. The Data Memory Usage for KATAN and KLEIN algorithm is equivalent.
	\end{itemize}
\end{frame}

\begin{frame}{With Respect To Memory Efficiency Cont'd}
	\begin{itemize}
    \begin{figure}
    \centering
    \includegraphics[width= 75 mm]{Pics/DM.png} 
    \caption{Memory Efficiency comparison of focused ciphers}
    \label{fig:6distinguisher}
\end{figure} 
	\end{itemize}
\end{frame}

\begin{frame}{With Respect To Security.}
	\begin{itemize}
		\item Taking degree of confusion and diffusion as the security criteria. KLEIN has least de-gree of diffusion highest degree of confusion.This is related to the fact that of differentstructure.KLEIN is SPN and others Feistel.
	\end{itemize}
\end{frame}

\begin{frame}{With Respect To Security Cont'd}
	\begin{itemize}
    \begin{figure}
    \centering
    \includegraphics[width= 75 mm]{Pics/DF.png} 
    \caption{THE ANALYSIS OF DIFFUSION}
    \label{fig:6distinguisher}
\end{figure} 
    \begin{figure}
    \centering
    \includegraphics[width= 75 mm]{Pics/DC.png} 
    \caption{THE ANALYSIS OF CONFUSION}
    \label{fig:6distinguisher}
\end{figure} 
	\end{itemize}
\end{frame}





