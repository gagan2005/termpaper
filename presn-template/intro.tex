\section{Introduction}

\begin{frame}{Introduction}
\begin{enumerate}
    \item As the development and usage of wireless computing and embedded systems is increasing we are being increasingly dependent on Ubiquitous computing examples are sensors, RFID tags etc.
    \item On these limited resource systems the selection of the security algorithms must be done carefully by taking the implementation costs along with the level of security provided into consideration.
    \item Many algorithms with various design strategies were proposed.Few of them were skipjack, KATAN, KTANTAN, PRESENT etc.
    \item Before wide implementation of a security algorithm, it should be thoroughly analysed.
    \seti
\end{enumerate}
\end{frame}

\begin{frame}{Introduction}
\begin{enumerate}
    \conti
    \item As the result of these analysis an attack on 31 out of 32 rounds of skipjack based on impossible differential is discovered. Also there are weak key attacks and linear attacks on PRESENT.
    \item  KLEIN is a lightweight block cipher which is mainly invented for devices like sensors which have very less resources.
    \item KLEIN is based on Substitution Permutation Networks.
    \seti
\end{enumerate}
\end{frame}